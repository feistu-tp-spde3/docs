\documentclass[a4paper,12pt]{article}
\usepackage[top=1in, bottom=1in, left=1in, right=1in]{geometry}
\renewcommand{\baselinestretch}{1.0}
\usepackage[utf8]{inputenc}
\usepackage[T1]{fontenc}
\usepackage[slovak]{babel}
\usepackage[pdftex]{graphicx}
\usepackage{caption}
\usepackage{url}
\usepackage{amsmath}
\usepackage{amssymb}
\usepackage{epsfig}
\usepackage{float}
\usepackage{lmodern}
\usepackage{breakurl}
\usepackage{chapterbib}
%\usepackage{hyperref}
\graphicspath{ {images/} }


\begin{document}
\begin{titlepage}
	\centering
	
	{\bfseries SLOVENSKÁ TECHNICKÁ UNIVERZITA  V BRATISLAVE\par}
	{\bfseries FAKULTA ELEKTROTECHNIKY A INFORMATIKY\par}
	\vspace{8cm}
	{\bfseries SYSTÉM PRE DETEKCIU EXPLOITOV A BEZPEČNOSTNÝCH INCIDENTOV ZO SIEŤOVEJ PREVÁDZKY 3\par}
	\vspace{0.5cm}
	{\bfseries TÍMOVÝ PROJEKT\par}
	\vspace{10cm}
	{\bfseries 2018 	\hfill \bfseries {Bc. Ivana Jozeková}} \\
	{ \hfill \bfseries {Bc. Patrik Kadlčík}} \\
	{ \hfill \bfseries {Bc. Matej Lovász}} \\
	{ \hfill \bfseries {Bc. Michal Malík}} \\ 
	{ \hfill \bfseries {Bc. Peter Malo}} \\
	{ \hfill \bfseries {Bc. Martin Martiška}}
	
\end{titlepage}

\newpage

\begin{titlepage}
	\centering
	
	{\bfseries SLOVENSKÁ TECHNICKÁ UNIVERZITA  V BRATISLAVE\par}
	{\bfseries FAKULTA ELEKTROTECHNIKY A INFORMATIKY\par}
	\vspace{8cm}
	{\bfseries SYSTÉM PRE DETEKCIU EXPLOITOV A BEZPEČNOSTNÝCH INCIDENTOV ZO SIEŤOVEJ PREVÁDZKY 3\par}
	\vspace{0.5cm}
	{\bfseries TÍMOVÝ PROJEKT\par}
	
	\vspace{\fill}
	\begin{tabbing}
		\hspace*{5cm}\= \kill
		Študijný program: \> Aplikovaná informatika \\
		Číslo študijného odboru: \> 2511 \\
		Názov študijného odboru: \> 9.2.9 Aplikovaná informatika \\
		Školiace pracovisko: \> Ústav informatiky a matematiky \\
		Vedúci záverečnej práce: \> Ing. Štefan Balogh, PhD. \\
	\end{tabbing}
	\vspace{\fill}
	
	{\bfseries Bratislava 2018 	\hfill \bfseries {Bc. Ivana Jozeková}} \\
	{ \hfill \bfseries {Bc. Patrik Kadlčík}} \\
	{ \hfill \bfseries {Bc. Matej Lovász}} \\
	{ \hfill \bfseries {Bc. Michal Malík}} \\ 
	{ \hfill \bfseries {Bc. Peter Malo}} \\
	{ \hfill \bfseries {Bc. Martin Martiška}}
	
\end{titlepage}
\newpage

\section*{SÚHRN}
\pagestyle{empty}
SLOVENSKÁ TECHNICKÁ UNIVERZITA \\
FAKULTA ELEKTROTECHNIKY A INFORMATIKY 

\begin{tabbing} 
	\hspace*{7cm}\= \kill
	Študijný program:\> Aplikovaná informatika \\
	Autor:\> Bc. Ivana Jozeková \\
			\> Bc. Patrik Kadlčík \\
			\> Bc. Matej Lovász \\
			\> Bc. Michal Malík \\
			\> Bc. Peter Malo \\
			\> Bc. Martin Martiška \\
	Vedúci záverečnej práce:\> Ing. Štefan Balogh, PhD. \\
	Miesto a rok predloženia práce:\> Bratislava 2019
\end{tabbing}

[text]
\newline \newline
Kľúčové slová: [text]
\newpage

\renewcommand{\contentsname}{Obsah}
\tableofcontents
\newpage
\renewcommand{\listfigurename}{Zoznam obrázkov}
\listoffigures
%\listoftables
\newpage


\section*{Úvod}
\setcounter{page}{1}
\pagestyle{plain}
\addcontentsline{toc}{section}{Úvod}
\newpage

\section{Ponuka}
\subsection{Riešiteľský kolektív}
\textbf{Bc. Ivana Jozeková} \\
\textbf{Pozícia v tíme: } \\ \\
Absolventka bakalárskeho štúdia na FEI STU v Bratislave, v študijnom programe Aplikovaná informatika, odbor Bezpečnosť informačných systémov. Bakalárske štúdium ukončila vypracovaním bakalárskej práce s názvom Vlastné čísla a Geršgorinove kruhy. \\ 

\noindent \textbf{Bc. Patrik Kadlčík} \\
\textbf{Pozícia v tíme: } \\ \\
Absolvent bakalárskeho štúdia na FEI STU v Bratislave, v študijnom programe Aplikovaná informatika, odbor Bezpečnosť informačných systémov. Bakalárske štúdium ukončil vypracovaním bakalárskej práce s názvom Bezpečnosť informačných systémov vo forme interaktívnej digitálnej hry. \\ 

\noindent \textbf{Bc. Matej Lovász} \\
\textbf{Pozícia v tíme: } \\ \\
Absolvent bakalárskeho štúdia na FEI STU v Bratislave, v študijnom programe Aplikovaná informatika, odbor Modelovanie a simulácia udalostných systémov. Bakalárske štúdium ukončil vypracovaním bakalárskej práce s názvom Pridávanie kreatívnych grafických elementov v reálnom čase na zobrazenie tváre. \\ 

\noindent \textbf{Bc. Michal Malík} \\
\textbf{Pozícia v tíme: } vedúci tímu \\ \\
Absolvent bakalárskeho štúdia na FEI STU v Bratislave, v študijnom programe Aplikovaná informatika, odbor Bezpečnosť informačných systémov. Bakalárske štúdium ukončil vypracovaním bakalárskej práce s názvom Návrh honeypotu s prvkami inteligencie. \\ 

\noindent \textbf{Bc. Peter Malo} \\
\textbf{Pozícia v tíme: } \\ \\
Absolvent bakalárskeho štúdia na FEI STU v Bratislave, v študijnom programe Aplikovaná informatika, odbor Bezpečnosť informačných systémov. Bakalárske štúdium ukončil vypracovaním bakalárskej práce s názvom Experimentálne porovnanie náhodne generovaných MQ-rovníc s rovnicami MQ-kryptosystémov. \\ 

\noindent \textbf{Bc. Martin Martiška} \\
\textbf{Pozícia v tíme: }\\ \\
Absolvent bakalárskeho štúdia na FEI STU v Bratislave, v študijnom programe Aplikovaná informatika, odbor Modelovanie a simulácia udalostných systémov. Bakalárske štúdium ukončil vypracovaním bakalárskej práce s názvom Porovnanie knižníc pre detekciu tváre pre OS Android. \\ 

\subsection{Anotácia tímového projektu}
Úlohou je doplniť a rozšíriť systém pre sledovanie sieťovej komunikácie detekcie útokov o nové funkcionality a metódy detekcie vybraných typov útokov. Systém bol vytvorený v predošlých tímových projektoch a je založený na známych metódach detekcie a využití databázy vzorov útokov. Analýza prebieha na serveri nad dátami získanými od ostaných počítačov (sieťové dáta, informácie z OS). Dáta sa ukladajú a spracovávajú na distribuovanom systéme (hadoop). Výsledkom analýzy je identifikácia, či ide o napadnutie systému. \\

Úlohy vyplývajúce zo zadania: 
\begin{itemize}
	\item Optimalizovať zber a spracovanie dát z klientských počítačov 
	\item Testovanie možnosti využitia existujúcich vzorov  \item sieťových útokov zo známych detekčných systémov 
	\item Analyzovať existujúce a použité metódy a vybrať novú metódu pre detekciu 
	\item Navrhnúť a implementovať nový modul pre detekciu exploitovania a sieťového útoku \\
\end{itemize}

\subsection{Motivácia}
\subsection{Organizácia tímového projektu}
\newpage

\section{Prehľad systému}
\newpage
\section{Agent}
\subsection{Windows}

\subsection{Linux}
\newpage

\section{Databáza}
\newpage

\section{TISMA systém}
\newpage

\section{Výsledky}
\newpage

\section*{Záver}
\addcontentsline{toc}{section}{Záver}
\newpage

\section*{Zoznam použitej literatúry}
\addcontentsline{toc}{section}{Zoznam použitej literatúry}

\newpage

\section*{Prílohy}
\pagenumbering{Roman}
\addcontentsline{toc}{section}{Prílohy}

\newpage
\end{document}
