\documentclass[a4paper,12pt]{article}
\usepackage[top=1in, bottom=1in, left=1in, right=1in]{geometry}
\renewcommand{\baselinestretch}{1.0}
\usepackage[utf8]{inputenc}
\usepackage[T1]{fontenc}
\usepackage[slovak]{babel}
\usepackage[pdftex]{graphicx}
\usepackage{caption}
\usepackage{url}
\usepackage{amsmath}
\usepackage{amssymb}
\usepackage{epsfig}
\usepackage{float}
\usepackage{lmodern}
\usepackage{breakurl}
\usepackage{chapterbib}
%\usepackage{hyperref}
\graphicspath{ {images/} }


\begin{document}
\begin{titlepage}
	\centering
	
	{\bfseries SLOVENSKÁ TECHNICKÁ UNIVERZITA  V BRATISLAVE\par}
	{\bfseries FAKULTA ELEKTROTECHNIKY A INFORMATIKY\par}
	\vspace{8cm}
	{\bfseries SYSTÉM PRE DETEKCIU EXPLOITOV A BEZPEČNOSTNÝCH INCIDENTOV ZO SIEŤOVEJ PREVÁDZKY 3\par}
	\vspace{0.5cm}
	{\bfseries TÍMOVÝ PROJEKT\par}
	\vspace{10cm}
	{\bfseries 2018 	\hfill \bfseries {Bc. Ivana Jozeková}} \\
	{ \hfill \bfseries {Bc. Patrik Kadlčík}} \\
	{ \hfill \bfseries {Bc. Matej Lovász}} \\
	{ \hfill \bfseries {Bc. Michal Malík}} \\ 
	{ \hfill \bfseries {Bc. Peter Malo}} \\
	{ \hfill \bfseries {Bc. Martin Martiška}}
	
\end{titlepage}

\newpage

\begin{titlepage}
	\centering
	
	{\bfseries SLOVENSKÁ TECHNICKÁ UNIVERZITA  V BRATISLAVE\par}
	{\bfseries FAKULTA ELEKTROTECHNIKY A INFORMATIKY\par}
	\vspace{8cm}
	{\bfseries SYSTÉM PRE DETEKCIU EXPLOITOV A BEZPEČNOSTNÝCH INCIDENTOV ZO SIEŤOVEJ PREVÁDZKY 3\par}
	\vspace{0.5cm}
	{\bfseries TÍMOVÝ PROJEKT\par}
	
	\vspace{\fill}
	\begin{tabbing}
		\hspace*{5cm}\= \kill
		Študijný program: \> Aplikovaná informatika \\
		Číslo študijného odboru: \> 2511 \\
		Názov študijného odboru: \> 9.2.9 Aplikovaná informatika \\
		Školiace pracovisko: \> Ústav informatiky a matematiky \\
		Vedúci záverečnej práce: \> Ing. Štefan Balogh, PhD. \\
	\end{tabbing}
	\vspace{\fill}
	
	{\bfseries Bratislava 2018 	\hfill \bfseries {Bc. Ivana Jozeková}} \\
	{ \hfill \bfseries {Bc. Patrik Kadlčík}} \\
	{ \hfill \bfseries {Bc. Matej Lovász}} \\
	{ \hfill \bfseries {Bc. Michal Malík}} \\ 
	{ \hfill \bfseries {Bc. Peter Malo}} \\
	{ \hfill \bfseries {Bc. Martin Martiška}}
	
\end{titlepage}
\newpage

\section*{SÚHRN}
\pagestyle{empty}
SLOVENSKÁ TECHNICKÁ UNIVERZITA \\
FAKULTA ELEKTROTECHNIKY A INFORMATIKY 

\begin{tabbing} 
	\hspace*{7cm}\= \kill
	Študijný program:\> Aplikovaná informatika \\
	Autor:\> Bc. Ivana Jozeková \\
			\> Bc. Patrik Kadlčík \\
			\> Bc. Matej Lovász \\
			\> Bc. Michal Malík \\
			\> Bc. Peter Malo \\
			\> Bc. Martin Martiška \\
	Vedúci záverečnej práce:\> Ing. Štefan Balogh, PhD. \\
	Miesto a rok predloženia práce:\> Bratislava 2019
\end{tabbing}

[text]
\newline \newline
Kľúčové slová: [text]
\newpage

\renewcommand{\contentsname}{Obsah}
\tableofcontents
\newpage
\renewcommand{\listfigurename}{Zoznam obrázkov}
\listoffigures
%\listoftables
\newpage


\section*{Úvod}
\setcounter{page}{1}
\pagestyle{plain}
\addcontentsline{toc}{section}{Úvod}
\newpage

\section{Ponuka}
\subsection{Riešiteľský kolektív}
\textbf{Bc. Ivana Jozeková} \\
\textbf{Pozícia v tíme: } \\ \\
Absolventka bakalárskeho štúdia na FEI STU v Bratislave, v študijnom programe Aplikovaná informatika, odbor Bezpečnosť informačných systémov. Bakalárske štúdium ukončila vypracovaním bakalárskej práce s názvom Vlastné čísla a Geršgorinove kruhy. \\ 

\noindent \textbf{Bc. Patrik Kadlčík} \\
\textbf{Pozícia v tíme: } \\ \\
Absolvent bakalárskeho štúdia na FEI STU v Bratislave, v študijnom programe Aplikovaná informatika, odbor Bezpečnosť informačných systémov. Bakalárske štúdium ukončil vypracovaním bakalárskej práce s názvom Bezpečnosť informačných systémov vo forme interaktívnej digitálnej hry. \\ 

\noindent \textbf{Bc. Matej Lovász} \\
\textbf{Pozícia v tíme: } \\ \\
Absolvent bakalárskeho štúdia na FEI STU v Bratislave, v študijnom programe Aplikovaná informatika, odbor Modelovanie a simulácia udalostných systémov. Bakalárske štúdium ukončil vypracovaním bakalárskej práce s názvom Pridávanie kreatívnych grafických elementov v reálnom čase na zobrazenie tváre. \\ 

\noindent \textbf{Bc. Michal Malík} \\
\textbf{Pozícia v tíme: } vedúci tímu \\ \\
Absolvent bakalárskeho štúdia na FEI STU v Bratislave, v študijnom programe Aplikovaná informatika, odbor Bezpečnosť informačných systémov. Bakalárske štúdium ukončil vypracovaním bakalárskej práce s názvom Návrh honeypotu s prvkami inteligencie. \\ 

\noindent \textbf{Bc. Peter Malo} \\
\textbf{Pozícia v tíme: } \\ \\
Absolvent bakalárskeho štúdia na FEI STU v Bratislave, v študijnom programe Aplikovaná informatika, odbor Bezpečnosť informačných systémov. Bakalárske štúdium ukončil vypracovaním bakalárskej práce s názvom Experimentálne porovnanie náhodne generovaných MQ-rovníc s rovnicami MQ-kryptosystémov. \\ 

\noindent \textbf{Bc. Martin Martiška} \\
\textbf{Pozícia v tíme: }\\ \\
Absolvent bakalárskeho štúdia na FEI STU v Bratislave, v študijnom programe Aplikovaná informatika, odbor Modelovanie a simulácia udalostných systémov. Bakalárske štúdium ukončil vypracovaním bakalárskej práce s názvom Porovnanie knižníc pre detekciu tváre pre OS Android. \\ 

\subsection{Anotácia tímového projektu}
Úlohou je doplniť a rozšíriť systém pre sledovanie sieťovej komunikácie detekcie útokov o nové funkcionality a metódy detekcie vybraných typov útokov. Systém bol vytvorený v predošlých tímových projektoch a je založený na známych metódach detekcie a využití databázy vzorov útokov. Analýza prebieha na serveri nad dátami získanými od ostaných počítačov (sieťové dáta, informácie z OS). Dáta sa ukladajú a spracovávajú na distribuovanom systéme (hadoop). Výsledkom analýzy je identifikácia, či ide o napadnutie systému. \\

Úlohy vyplývajúce zo zadania: 
\begin{itemize}
	\item Optimalizovať zber a spracovanie dát z klientských počítačov 
	\item Testovanie možnosti využitia existujúcich vzorov  \item sieťových útokov zo známych detekčných systémov 
	\item Analyzovať existujúce a použité metódy a vybrať novú metódu pre detekciu 
	\item Navrhnúť a implementovať nový modul pre detekciu exploitovania a sieťového útoku \\
\end{itemize}

\subsection{Motivácia}
\subsection{Organizácia tímového projektu}
\newpage

\section{Prehľad systému}
\newpage
\section{Agent}
\subsection{Windows}

\subsection{Linux}
\newpage

\section{Databáza}
\newpage

\section{TISMA systém}
\newpage

\section{Výsledky}
\newpage

%-------------------------------- TODO: Zaradit do textu, skontrolovat pravopis, stylistiku a upravit formatovanie
\section{Kompilácia a prostredie}

\subsection{Agent}

Kompilácia agenta bola pozmenená, keďže sa pridali nové knižnice a zjednotil sa zdrojový kód pre obe platformy (Windows aj Linux).

\subsubsection{Windows}

Na Windowse je ku kompilácií potrebné Microsoft Visual Studio 2017 (v141).
Potrebné sú knižnice boost (verzia 1.68.0), Npcap a NPCAP SDK. Je potrebné ich rozbaliť  a nainštalovať (na ľubovoľné miesto na disku).  Pri knižnici Npcap je potreba zvoliť „Winpcap compact mode“. Ak už máme knižnice pripravené, pridáme do systémových premenných prostredia hodnoty BOOST\_INCLUDE\_PATH, BOOST\_LIB\_PATH, NPCAP\_INCLUDE\_PATH, NPCAP\_LIB\_PATH a nastavíme ich, aby ukazovali na include alebo library cestu danej knižnice.
\\
Príklad: \\
BOOST\_INCLUDE\_PATH=C:$\textbackslash$boost\_1\_68\_0\_32 \\
BOOST\_LIB\_PATH=C:$\textbackslash$boost\_1\_68\_0\_32$\textbackslash$lib32-msvc-14.1 \\
NPCAP\_INCLUDE\_PATH=C:$\textbackslash$npcap$\textbackslash$Include \\
NPCAP\_LIB\_PATH=C:$\textbackslash$npcap$\textbackslash$Lib \\
\\
Na kompilovanie treba otvoriť .sln súbor, ktorý sa nachádza medzi súbormi projektu, pomocou Visual Studia. V pravo máme okno „Solution Explorer“, v ktorom máme položku projektu („Agent“). V nastaveniach projektu je potrebné pridať knižnice, ktoré využívame. Navigujeme:
Right-click „Agent“ $\rightarrow$ Properties $\rightarrow$ Configuration Properties $\rightarrow$ VC++ Directories.
Pred tým, než začneme pridávať nastavenia je potrebné sa uistiť, že máme v hornej lište nastavené políčko “Configuration” na “All Configurations” a “Platform” na “Win32”. Pridáme do tabuľky „General“ v riadku pre „Include directories“ cesty pre include našich knižníc - \$(BOOST\_INCLUDE\_PATH);\$(NPCAP\_INCLUDE\_PATH);  a do „Library directories“ \$(BOOST\_LIB\_PATH);\$(NPCAP\_LIB\_PATH);. 

\subsubsection{Linux}

Na Linuxe je potreba kompilátora gcc (verzia 5.1 a vyššia), rovnako ako pri Windowse nainštalujeme knižnice boost (ľubovoľný spôsob inštalácie) a libpcap pomocou príkazu sudo apt-get install automake libpcap-dev build-essential.
Ak sme boost kompilovali sami, treba v súbore CMakeLists.txt zmeniť set (BOOST\_ROOT "/home/user/boost\_1\_68\_0") tak, aby ukazoval na miesto, kde sme zdrojové súbory vykompilovali. Ak sme boost nekompilovali sami, tak zmeníme v tom istom súbore hodnotu set (Boost\_NO\_SYSTEM\_PATHS TRUE) na FALSE.
\\
Po vykonaní všetkých týchto inštrukcií, stačí použiť príkazy „cmake .“ a následne „make“ v adresári so súborom CMakeLists.txt, čím sa nám zdrojový kód skompiluje.
\\
\\
\textbf{CMake} \\
%-------------------- cmake treba naformatovat do kodoveho stylu
Pridané zdrojové súbory (zvlášť .hpp a .cpp) \\
file (GLOB CPP\_FILES src/*.cpp) \\
file (GLOB HPP\_FILES src/*.hpp) \\
\\
set (SOURCE\_FILES \$\{CPP\_FILES\} \$\{HPP\_FILES\}) \\
\\
Nastavenie libpcap symbolového súboru \\
if (EXISTS /usr/lib/x86\_64-linux-gnu/libpcap.so) \\
set (LIBS /usr/lib/x86\_64-linux-gnu/libpcap.so) \\
else () \\
set (LIBS /usr/lib/i386-linux-gnu/libpcap.so) \\
endif () \\
\\
Nastavenie boostu  \\
set (Boost\_USE\_STATIC\_LIBS OFF) \\
set (Boost\_USE\_MULTITHREADED ON) \\
set (Boost\_USE\_STATIC\_RUNTIME OFF) \\ \\
find\_package(Boost 1.68.0 COMPONENTS system filesystem chrono thread) \\ \\
Na konci sa naviažu zdrojové súbory \\
add\_executable(\$\{TARGET\_NAME\} \$\{SOURCE\_FILES\}) \\
Nalinkujú sa knižnice  \\
target\_link\_libraries(\$\{TARGET\_NAME\} \$\{LIBS\} \$\{Boost\_LIBRARIES\}) \\
A nazáver sa nastaví kompilácia \\
target\_compile\_options(\$\{TARGET\_NAME\} PRIVATE -std=c++11) \\

\subsection{Monitor}

Potreba Microsoft Visual Studia (Windows), gcc (Linux) a knižnice boost (viď sekcia kompilácie Agenta) a navyše je potrebný MySQL Connector pre C++ (verzia 8.0).

\subsubsection{Windows}

Na Windowse vytvoríme pre cesty k include a library systémové premenné prostredia MYSQL\_CONNECTOR\_INCLUDE\_PATH a MYSQL\_CONNECTOR\_LIB\_PATH. \\ \\
Príklad: \\
MYSQL\_CONNECTOR\_INCLUDE\_PATH=C:$\textbackslash$mysql-connector-c++-8.0.13-win32$\textbackslash$include \\
MYSQL\_CONNECTOR\_LIB\_PATH=C:$\textbackslash$mysql-connector-c++-8.0.13-win32$\textbackslash$lib$\textbackslash$vs14 \\ \\
Tieto hodnoty pridáme do nastavení projektu (podobne ako v časti pre Agenta) - Right-click project $\rightarrow$ Properties $\rightarrow$ Configuration Properties $\rightarrow$ VC++ Directories, kde pridáme do „Include Directories“ \$(MYSQL\_CONNECTOR\_INCLUDE\_PATH); a do “Library directories” \$(MYSQL\_CONNECTOR\_LIB\_PATH);. Následne navigujeme od Configuration Properties -> Linker -> General a pridáme do poľa „Additional Library Dependencies“ hodnotu \$(MYSQL\_CONNECTOR\_LIBRARY\_PATH);. Prepneme cez Linker $\rightarrow$ Input a do “Additional Dependencies” pridáme názov knižnice „mysqlcppconn.lib“. \\ 
Na to aby sme mohli program spúšťať, musíme do súboru s vygenerovaným .exe súborom ešte pridať knižnice mysqlcppconn-7-vs14.dll, ssleay32.dll, libeay32.dll (z MySQL Connectora).

\subsubsection{Linux}

Na Linuxe je postup identický s postupom kompilácie agenta, akurát je potreba doinštalovať spomínaný Connector pre MySQL (Poznámka: Ak používame Debian, nesmie sa používať inštalácia z repozitára!). Kompiláciu stačí spustiť pomocou príkazov „cmake .“, potom „make“. \\ \\
\textbf{CMake} \\ \\
Veľmi podobná štruktúra ako pri Agentovi – rovnaké nastavenie boostu, vyhľadávanie komponentov, až na symbolový súbor pre mysql connector.\\ \\
if (EXISTS /usr/lib/x86\_64-linux-gnu/libmysqlcppconn.so) \\
set (LIBS /usr/lib/x86\_64-linux-gnu/libmysqlcppconn.so) \\
else () \\
set (LIBS /usr/lib/i386-linux-gnu/libmysqlcppconn.so) \\
endif () \\




\section*{Záver}
\addcontentsline{toc}{section}{Záver}
\newpage

\section*{Zoznam použitej literatúry}
\addcontentsline{toc}{section}{Zoznam použitej literatúry}

\newpage

\section*{Prílohy}
\pagenumbering{Roman}
\addcontentsline{toc}{section}{Prílohy}

\newpage
\end{document}
